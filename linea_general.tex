\documentclass[letterpaper,12pt]{article}

\usepackage[utf8]{inputenc}
\usepackage[spanish]{babel}
\usepackage{fontenc}
\usepackage[pdftex]{graphicx}


\title{Resumen general de la propuesta}
\author{Julio César Enciso Alva}

\begin{document}
	
	\maketitle
	
	The Malevolent Hero es un proyecto de cómic experimental, una suerte de pequeño
	multiverso marvelita a la medida: siete historias, cada una única y con un propósito claro
	dentro del marco.
	
	Los protagonistas, cada cual con sus intereses y motivaciones, deben sortear desproporcionadas
	amenazas a sus
	respectivos mundos; aunque no lo saben, todas ellas son provocadas por The Malevolent Hero
	-cuya naturaleza o identidad es aun un secreto.
	
	Lo que su nombre quiere significar es que esta obra tiene como objetivo presentar una 
	mirada interesante sobre la lucha entre el bien y el mal: no sabemos quienes están de qué lado.
	
	Y es que entre las estructuras héroe-villano de cada historia, distintos personajes defienden
	una multitud de posturas desde una gran variedad de puntos de vista; bajo la misma 
	premisa una persona puede ser un héroe en una historia y un villano en otra.
	
	Así es la moraleja de The Malevolent Hero, y bajo aquella existirá.
	
	\newpage
	
	\section*{Zeine, el vengador}
	
	\subsection*{La justicia de los iracundos}
	
	Medieval con magos, hadas, demonios. El protagonista busca venganza.
	
	\subsection*{Sinopsis}
	
	Nos hallamos en un reino medieval donde la fuerza más grande conocido es un tipo de cristales mágicos, cuyo
	manejo es un arte conocido por pocos.
	
	Un mundo donde las personas conviven diariamente con \textit{hadas} y \textit{demonios}
	-personificaciones humanoides de los defectos y las virtudes-, entre otras criaturas fantásticas. 
	
	Una guerra estalla tras ser robado un cristal con poder infinito, y el clan de los demonios es exterminado
	tras ser acusados unánimemente.
	
	Nosotros historia sigue aquél cuyo objetivo es encontrar al verdadero cupable: Zeine, el demonio de la
	venganza.
	
	\newpage
	
	\section*{Digital Girl Kannela}
	
	\subsection*{Las dudas de los inocentes}
	
	Chica mágica en un mundo digital, en Alemania del Este.
	
	\section*{Sinopsis}
	
	Kanela es una chica de 13 años muy tímida, con pocos amigos, pero con gran afición por la computación. 
	
	Como pasatiempos, un día investiga la leyenda urbana sobre cierta transmisión misteriosa, sólo para
	encontrar que es entrada a un \textit{Mundo Virtual}. Una vez allí se hace amiga de un pequeño dragón
	llamado Leswar, quien la guía por su fantástico mundo.
	
	Pronto se dan cuenta que Kanela resulta ser la primera persona con un
	\textit{corazón guerrero} en el Mundo Virtual, convirtiéndose en el único ser que puede luchar 
	contra un malvado emperador malvado.
	
	Leswar y Kanela son ahora compañeros en una lucha que se extiende por ambos mundos.
	
	\newpage
	
	\section*{Sector 57}
	
	\section*{La patria de los infelices}
	
	Serie policiaca en mundo de genios. Protagonista hijo de un malvado.
	
	\subsubsection*{}
	
	\newpage
	
	\section*{Reacción encadenada}
	
	\subsection*{Los anhelos de los moribundos}
	
	Reinterpretación de la conquista: dioses reales, pueblos en guerra.
	
	\newpage
	
	\section*{Historia sin título sobre el amor y la identidad propia}
	
	\subsection*{Las lágrimas de los dioses}
	
	Shojo con ventana entre universos paralelos con géneros cambiados.
	
	\section*{Sinopsis}
	
	\{gurdaespacio\} es un chico normal, con una vida normal y algunos amigos.
	Va a la secundaria y juega basquetball. Tiene dos hermanos menores y
	está pretendiendo a una chica.
	
	Un día, descubre entre entre las antigüedades familiares un reloj de bolsillo;
	resulta que en la carátula tiene una ventan hacia un universo paralelo.
	
	En el otro universo todo es exactamente igual salvo precisamente él... o ella,
	y es que su contraparte en aquél mundo es una chica, tan o más normal que él 
	y con mucho en común.
	
	Así, esta historia narra el día a día de \{gurdaespacio\} y su versión alterna
	femenina, y cómo usará la conexión especial con su \textit{otro yo} para
	descubrir cosas sobre sí mismo y las personas que lo rodean.
	
	\newpage
	
	\section*{Robot Ragnarok}
	
	\subsection*{El consuelo de los miserables}	
	
	Imperio galáctico caído hace mucho con robots asimovianos: vestigios.
	
	\section*{Sinopsis}
	
	Fouri es una arqueóloga / saqueadora de tesoros, de entre los que abundan tras la caída
	del último imperio galáctico. 
	
	El poderío de aquella potencia se basó en 
	el orden utópico establecido por robots inteligentes, regidos a su vez
	por las tres leyes asimovianas:
	\begin{itemize}
		\item Un robot no puede dañar o, por su inacción, dejar
		que un ser humano sufra daño.
		\item Un robot debe obedecer las órdenes de un ser humanos, excepto
		en caso de que interfiera con la primera ley.
		\item Un robot debe proteger su propia existencia , excepto en caso
		de que interfiera con las primeras dos leyes.
	\end{itemize}
	
	Nuestra historia empieza cuando Fouri encuentra accidentalmente un secreto
	más allá de cualquier leyenda: dos armaduras-robóticas gemelas,
	poderosas como ninguna otra, limitadas por las tres leyes pero obedientes a la 
	\textit{ley cero}:
	\begin{itemize}
		\item Un robot debe proteger a la humanidad en conjunto, aún si al hacerlo
		desobedece las tres primeras leyes.
	\end{itemize}
	
	¿Podrán Fouri y su armadura evitar la destrucción del universo a manos de la segunda armadura?
	
	\newpage
	
	\section*{Another Faust}
	
	\subsection*{El silencio de los culpables}
	
	Empresario en la cima hace contrato con Nick McCore, comerciante de almas.
	
	\section*{Sinopsis}
	
	Thomas Orrow es un magnate de los negocios, dueño de la cadena televisiva más
	grande del mundo y de muchas otras que ha comprado con el tiempo. Es joven y rico,
	exitoso con las mujeres y popular en sociedad.
	
	Pero sin embargo, no está satisfecho.
	
	Es bajo estas circunstancias que conoce a Nicholas McCore, dueño de una televisora
	local y una cadena productora de carne; Nick le propone un contrato muy
	especial, en el cual Tom estará en condición de pedirle cualquier deseo que
	le plazca a cambio de hacer ciertos \textit{trabajos} como compensación.
	
	Aunque al principio le parece una broma, Tom se convence de que Nick McCore es
	algo más que humano...
	
	\newpage
	
	\section*{Dema no Miniserie}
	
	\subsection*{Los secretos del sinsentido}
	
	Fanservice de relleno con Dema. Resulta que es muy importante.
	
	\subsection*{Sinopsis}
	
	Dema es un ente metaficticio en el cuerpo -y la mente- de una adolescente,
	cuya meta en la vida es ser un personaje popular y querido por el público.
	
	Junto con Dshé y D', son representaciones antropomórficas de The Malevolent Hero
	en sus distintas \textit{partes}: emocional, racional y moral.
	
	Entre sus aventuras, Dshé trata de componer una suerte de artículos de divulgación como
	trasfondo científico de la historia, mientras que Dema trae montones de fanservice
	e historias de relleno; D' se limita a ver y hacer revelaciones poco claras bajo
	la pregunta que, según ella, debiera ser el verdadero título de esta obra: 
	\textit{W... is The Malevolent Hero?}
	
\end{document}
